\chapter{PENDAHULUAN}

\section{Latar Belakang}

% Paragraf pertama
Dashboard kinerja adalah salah satu alat teknologi yang dapat membantu memberikan informasi bagi manajemen rumah sakit untuk
melakukan pengambilan keputusan. Dengan kemajuan teknologi informasi dalam industri kesehatan, penggunaan dashboard kinerja telah menjadi semakin penting.
Berdasarkan penelitian milik Luigi Jesus Basile, pemanfaatan dashboard kinerja dan Business Intelligence (BI) dalam pengambilan keputusan dapat mengungguli praktik berbasis pengalaman 
dalam mengelola proses di sektor kesehatan\parencite{Basile2023}. Hal Ini diperkuat oleh data statistik yang menunjukkan bahwa rumah sakit yang aktif 
menggunakan dashboard kinerja dan alat BI seringkali mencapai tingkat efisiensi yang lebih tinggi daripada mereka yang masih 
mengandalkan pengalaman semata. Implementasi dashboard kinerja menjadi lebih mendesak dalam berbagai situasi kesehatan yang memerlukan pemantauan 
real-time dan analisis data untuk membantu rumah sakit merespons dengan lebih cepat dan tepat.

% Paragraf kedua
Berdasarkan wawancara dengan narasumber dari Rumah Sakit Bhayangkara, didapatkan bahwa saat ini setiap cabang rumah sakit masih memiliki 
sistem informasi yang terpisah-pisah dan dijalankan secara lokal. Hal ini menyebabkan basis data dari setiap cabang rumah sakit belum terintegrasi dengan basis data sentral. 
Basis data sentral tersebut diharapkan terhubung dengan dashboard kinerja rumah sakit. Menurut data statistik yang diterbitkan oleh
Health IT Analytics, hanya sekitar 30\% rumah sakit di dunia yang memiliki sistem informasi terintegrasi sepenuhnya. 
Ketidaktersediaan integrasi data menjadi hambatan dalam membangun dashboard kinerja guna membantu proses manajemen rumah sakit.

%paragraf ketiga
Upaya yang telah dilakukan untuk mengintegrasiakan basis data tersebut adalah dengan melakukan migrasi data secara manual. 

%paragraf keempat
Upaya yang telah diberikan tersebut masih menimbulkan masalah, yaitu  tenaga dan waktu yang diperlukan untuk melakukan migrasi data secara manual sangat besar.
Menurut laporan yang diterbitkan oleh HealthIT.gov, proses migrasi data manual seringkali menghabiskan lebih dari 40\% waktu staf TI 
di rumah sakit. Hal ini dapat mengganggu fokus mereka pada tugas-tugas lain yang juga penting dalam pengelolaan teknologi 
informasi di rumah sakit. Selain itu ancaman kebocoran data juga menjadi masalah yang sering terjadi dalam proses migrasi data secara manual.
Data dari Ponemon Institute mengungkapkan bahwa biaya kebocoran data di sektor kesehatan dapat mencapai rata-rata 
\$7,1 juta per tahun. Ini menggarisbawahi pentingnya menjaga keamanan data pasien selama proses migrasi, 
yang dapat menjadi sangat kompleks dan rentan terhadap serangan siber.

%paragraf kelima
Dari permasalahan yang telah diuraikan di atas, diperlukan suatu sistem yang dapat mengintegrasikan basis data dari setiap cabang rumah sakit.
Pengembangan sistem integrasi data yang canggih dan aman dapat membantu rumah sakit seperti Rumah Sakit Bhayangkara dalam membangun dashboard kinerja
guna meningkatkan efisiensi operasional dan meningkatkan pelayanan kesehatan bagi pasien. Solusi ini diharapkan dapat membantu rumah sakit
bergerak menuju sistem informasi yang lebih terpadu dan efisien. 

\section{Rumusan Masalah}

% Ubah paragraf berikut sesuai dengan rumusan masalah dari tugas akhir
Berdasarkan hal yang telah dipaparkan di latar belakang, \lipsum[4]

\section{Batasan Masalah atau Ruang Lingkup}

\lipsum[6]

\section{Tujuan}

% Ubah paragraf berikut sesuai dengan tujuan penelitian dari tugas akhir
Tujuan dari penelitian ini adalah \lipsum[7][1-14]

\section{Manfaat}

% Ubah paragraf berikut sesuai dengan tujuan penelitian dari tugas akhir
Manfaat dari penelitian ini adalah \lipsum[8][1-14]
