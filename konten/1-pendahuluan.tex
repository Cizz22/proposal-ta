\chapter{PENDAHULUAN}

\section{Latar Belakang}

% Paragraf pertama
Dashboard kinerja adalah salah satu alat teknologi yang dapat membantu memberikan informasi bagi manajemen rumah sakit untuk
melakukan pengambilan keputusan.

% Paragraf kedua
Berdasarkan wawancara dengan narasumber dari RSUD Dr. Soetomo, didapatkan bahwa saat ini setiap cabang rumah sakit masih memiliki 
sistem informasi yang terpisah-pisah dan dijalankan secara lokal. Hal ini menyebabkan basis data dari setiap cabang rumah sakit belum terintegrasi dengan basis data sentral. 
Dashboard kinerja nantinya diharapkan terhubung dengan basis data sentral sehingga dapat memberikan informasi yang lebih akurat dan terintegrasi.

%paragraf ketiga
Upaya yang telah dilakukan untuk mengintegrasiakan basis data tersebut adalah dengan melakukan migrasi data secara manual.

%paragraf keempat
Upaya yang telah diberikan tersebut masih menimbulkan masalah, yaitu tenaga dan waktu yang diperlukan untuk melakukan migrasi data secara manual sangat besar serta adanya kerawanan dalam kebocoran data.

%paragraf kelima
Dari permasalahan yang telah diuraikan di atas, diperlukan suatu sistem yang dapat mengintegrasikan basis data dari setiap cabang rumah sakit secara otomatis dan tetap menjaga keamanan data.

\section{Rumusan Masalah}

% Ubah paragraf berikut sesuai dengan rumusan masalah dari tugas akhir
Berdasarkan hal yang telah dipaparkan di latar belakang, \lipsum[4]

\section{Batasan Masalah atau Ruang Lingkup}

\lipsum[6]

\section{Tujuan}

% Ubah paragraf berikut sesuai dengan tujuan penelitian dari tugas akhir
Tujuan dari penelitian ini adalah \lipsum[7][1-14]

\section{Manfaat}

% Ubah paragraf berikut sesuai dengan tujuan penelitian dari tugas akhir
Manfaat dari penelitian ini adalah \lipsum[8][1-14]
