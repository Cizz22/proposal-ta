\chapter{PENDAHULUAN}

\section{Latar Belakang}

% Paragraf pertama
Dashboard kinerja adalah salah satu alat teknologi yang dapat membantu memberikan informasi bagi manajemen rumah sakit untuk
melakukan pengambilan keputusan. Dengan kemajuan teknologi informasi dalam industri kesehatan, penggunaan dashboard kinerja telah menjadi semakin penting.
Berdasarkan penelitian milik Luigi Jesus Basile, pemanfaatan dashboard kinerja dan \emph{Business Intelligence}(BI) dalam pengambilan keputusan dapat mengungguli praktik berbasis pengalaman 
dalam mengelola proses di sektor kesehatan \parencite{Basile2023}. Selain itu, laporan yang dikeluarkan oleh Capital link terkait \emph{Performance Benchmarking Toolkit for Health Centers: Tracking Data to Improve Financial Performance } 
menjelaskan bahwa penerapan alat analis data data membantu pemimpin dalam melacak kinerja secara lebih efektif dan efisien, 
memahami faktor utama yang memengaruhi, serta menggabungkan pemahaman tentang operasional untuk menjadikan 
pusat kesehatan lebih berkelanjutan secara finansial dan mencapai kesuksesan yang berkelanjutan \parencite{CapitalLink2017}. 
Implementasi dashboard kinerja menjadi lebih mendesak dalam berbagai situasi kesehatan yang memerlukan pemantauan 
real-time dan analisis data untuk membantu rumah sakit merespons dengan lebih cepat dan tepat.

% Paragraf kedua
Berdasarkan wawancara dengan narasumber dari Rumah Sakit Bhayangkara, didapatkan bahwa saat ini setiap cabang rumah sakit masih memiliki 
sistem informasi yang terpisah-pisah dan dijalankan secara lokal. Hal ini menyebabkan basis data dari setiap cabang rumah sakit belum terintegrasi dengan basis data sentral. 
Basis data sentral tersebut diharapkan terhubung dengan dashboard kinerja rumah sakit guna memantau kinerja baik keseluruhan cabang ataupun satu-persatu.
Ketidaktersediaan integrasi data tersebut menjadi hambatan dalam membangun dashboard kinerja.

%paragraf ketiga
Upaya yang telah dilakukan untuk mengintegrasiakan data adalah dengan melakukan migrasi data secara manual. 
(Perlu wawancara lebih kepada pihak rumah sakit bhayangkara untuk menjelaskan lebih dalam terkait proses 
migrasi data secara manual)

%paragraf keempat
Upaya yang telah diberikan tersebut masih menimbulkan masalah, yaitu tenaga dan waktu yang diperlukan untuk melakukan migrasi data secara manual sangat besar. 
Sebagaimana yang telah dijelaskan oleh penelitian yang dilakukan oleh \textcite{Elamparithi2015}, migrasi 
sistem dapat menjadi tugas yang memakan waktu dan sangat mahal; oleh karena itu, organisasi perlu 
menyederhanakan proses migrasi dan menjadikannya semaksimal mungkin dalam hal efisiensi 
biaya. Selain itu ancaman kebocoran data juga menjadi masalah yang sering terjadi dalam proses migrasi data secara manual.
Data dari Ponemon Institute dalam laporan \emph{Cost of a Data Breach Report 2023 } mengungkapkan bahwa biaya kebocoran data di sektor 
kesehatan dapat mencapai rata-rata \$11 juta \parencite{Ponemon2023}. Hal ini menggarisbawahi pentingnya menjaga keamanan data pasien selama proses migrasi data, 
yang dapat menjadi sangat kompleks dan rentan terhadap serangan siber.

%paragraf kelima
Dari permasalahan yang telah diuraikan di atas, diperlukan suatu sistem yang dapat mengintegrasikan basis data dari setiap cabang rumah sakit ke basis data sentral
yang terhubung dengan dashboard kinerja. Pengembangan sistem integrasi data yang canggih dan aman diharapkan dapat membantu rumah sakit seperti Rumah Sakit Bhayangkara dalam 
pengembangan dashboard kinerja guna meningkatkan efisiensi operasional dan meningkatkan pelayanan kesehatan bagi pasien. 

\section{Rumusan Masalah}

\section{Batasan Masalah atau Ruang Lingkup}

\section{Tujuan}

\section{Manfaat}

